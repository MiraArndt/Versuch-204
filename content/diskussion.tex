\section{Diskussion}
\label{sec:Diskussion}

Bei der statistischen Methode konnten die erwarteten Zusammenhänge, dass
ein dünner Stab schlechter Wärme leitet als ein breiter Stab, nachgewiesen werden.
Es war auch zu erwarten, dass Edelstahl die Wärme nicht so gut leitet
wie Aluminium, was auch aus dem Versuch hervorgeht. Die Temperaturwerte aller Stäbe sollten sich einer
oberen Schranke annähern, doch die Graphen \ref{fig:a} und \ref{fig:b} zeigen eher ein asymptotisches 
Verhalten.

\noindent Die prozentualen Fehlerwerte der Messwerte zu den Literaturwerten der Wärmeleitfähigkeit sind in Tabelle \ref{tab:diskussion} aufgeführt.


\begin{table}[H]
\normalsize

\centering
\sisetup{table-format=4.0}
\begin{tabular}{c c c c}
\toprule
        Material & $\kappa_{Messung} \,/\, \si{\watt\per\meter\per\kelvin}$ & $ \kappa_{Literaturwert} $\cite{schweiz}$\,/\, \si{\watt\per\meter\per\kelvin}$ & Prozentualer Fehler$\,/\, \% $\\
        \midrule
        Messing      &   78   &   81   &   3,70\\
        Aluminium   &   210   &   220   &    4,55\\
        Edelstahl     &   15,5   &   20   &   37,5\\

\bottomrule

\end{tabular}

\caption{Vergleich der Messwerte und Literaturwerte der Wärmeleitfähigkeit}
\label{tab:diskussion}
\end{table}

\noindent Die Abweichungen lassen sich durch statistische Fehler bei der Messung erklären.
Dazu gehört die Ungenauigkeit der Periodizität des Temperaturanstiegs und Abfalls, da
das Umstellen der Kühlung und Erhitzung nach einer Stoppuhr erfolgte.
Außerdem kann trotz Isolierung nicht davon ausgegangen werden, dass Die Probenstäbe keine Temperatureinflüsse von außen Erfahren haben. 