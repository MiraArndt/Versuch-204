\section{Theorie}
\label{sec:Theorie}

Wenn sich die Temperatur innerhalb eines Körpers nicht in einem Gleichgewichtszustand
befindet, trifft ein Wärmetransport in Form von mindestens einer der drei Möglichkeiten 
Konvektion, Wärmestrahlung und Wärmeleitung auf.
Wird ein Stab, mit Länge $L$, dem Querschnitt $A$, der Dichte $\rho$ und der spezifischen 
Wärme $c$ so erhitzt, dass er zum Beispiel an einem Ende wärme ist, als am anderen, so 
fließt innerhalb der Zeit $\operatorname{d}t$ die Wärmemenge 
\begin{equation}
    \symup{d}Q = -\kappa A \frac{\partial T}{\partial x} \symup{d}t
\end{equation}
\noindent durch die Querschnittsfläche $A$. Das Minuszeichen kommt daher, dass Wärme immer
vom warmen ins kalte Reservoir fließt. Der Buchstabe $\kappa$ bezeichnet die
gesuchte Wärmeleitfähigkeit, welche materialabhängig ist. In dem vorliegenden eindimensionalen 
Fall lässt sich die Wärmestromdichte $j$ als
\begin{equation}
    j = - \kappa \frac{\partial T}{\partial x}
    \label{eq:waermestromdichte}
\end{equation}
\noindent definieren. Weiterhin lässt sich die Wärme innerhalb eines Körpers über \cite{sample2}
\begin{equation}
    Q = m c T 
    \label{eq:waerme}
\end{equation}
\noindent berechnen. Werden nun die Gleichungen \ref{eq:waermestromdichte} und \ref{eq:waerme}
in die Kontinuitätsgleichung 
\begin{equation}
    \frac{\frac{\partial Q}{\partial V}}{\partial t} + \nabla \vec{j} = 0
\end{equation}
\noindent eingesetzt (V steht für das Volumen), so ergibt sich die eindimensionale 
Wärmeleitungsgleichung
\begin{equation}
    \frac{\partial T}{\partial t} = \frac{\kappa}{\rho c} \frac{\partial^2 T}{\partial x^2}.
\end{equation}
\noindent Mithilfe dieser Gleichung lässt sich die räumliche und zeitliche Temperaturentwicklung 
in dem zu untersuchenden Köper beschreiben. Die Lösung ist abhängig von Anfangsbedingungen 
und der Stabgeometrie. Weiterhin wird die Temperaturleitfähigkeit, ein Maß
zur Angabe, wie schnell ein Temperaturausgleich zustande kommt, als
\begin{equation}
    \sigma_{T} = \frac{\kappa}{\rho c}
\end{equation}
\noindent definiert. Wird ein Stab über die Periode $T$ abwechselnd erhitzt und abgekühlt, 
ergibt sich die allgemeine Lösung
\begin{equation}
    T(x,t) = T_{max} e^{-\sqrt{\frac{\rho \omega c}{2\kappa}}\,x} cos \left( \omega t - \sqrt{\frac{\omega \rho c}{2 \kappa}}x \right).
\end{equation}
\noindent Diese Lösung entspricht einer Temperaturwelle, mit Amplitude $T_{max}$, welche sich mit der Phasengeschwindigkeit
\begin{equation}
    v = \frac{\omega}{k} = \sqrt{ \frac{2 \kappa \omega}{\rho c} }
\end{equation}
\noindent in dem Stab ausbreitet. Die Dämpfung dieser Welle lässt sich über das Verhältnis der
Amplituden $A_1$ und $A_2$ an den Stellen $x_1$ und $x_2$ bestimmen. Werden zusätzlich
$\omega = \frac{2 \pi}{T^*}$ ($T^*$ gibt die Periodendauer der Wärmewelle an) und 
$\Phi = \frac{2 \pi \Delta t}{T^*}$ verwendet, lässt die die Wärmeleitfähigkeit $\kappa$ 
über
\begin{equation}
    \kappa = \frac{\rho c (\Delta x)^2}{2 \Delta t \ln{\frac{A_1}{A_2}}}
\end{equation}
\noindent errechnet werden. Hierbei gilt 
\begin{align}
    \Delta x = x_2 - x_1 \nonumber \\
    \Delta t = t_2 - t_1 \nonumber.
\end{align}