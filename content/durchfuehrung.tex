\section{Durchführung}
\label{sec:Durchführung}

\subsection{Allgemein}

Bei dem Versuch sind vier Probestäbchen (zwei Messingstäbe mit unterschiedlichen 
Querschnittsflächen sowie jeweils ein Aluminiumstab und ein Edelstahlstab mit gleichen 
Querschnittsflächen) auf einer Platte befestigt. Wie der Abbildung \ref{fig:durchfuehrungbild1}
entnommen werden kann, werden die Stäbe an einer Seite mithilfe eines Peltierelements 
simultan erhitzt beziehungsweise abgekühlt. An jedem Stab befinden sich zwei Thermoelemente,
mit denen die Temperatur an dem Stab gemessen werden kann. Die Grundplatte ist zusätzlich mit 
dem 'XPlorer GLX' (Datenlogger) verbunden, welcher die aufgenommenen
Daten während der Messung anzeigen, aber auch abspeichern kann. Während der Messung soll die 
Wärmeisolierung über die Stäbe gelegt werden, damit die Abgabe von Wärme an die Umgebung so 
gering die möglich gehalten werden kann. 


\begin{figure}[H]
    \centering
    \includegraphics[width=13cm]{content/durchfuehrung1.png}
    \label{fig:durchfuehrungbild1}
    \caption{NAMENAMENAMENAME}
\end{figure}


\subsection{Statische Methode}

Zunächst soll der Abstand der Thermoelemente eines Stabes voneinander gemessen werden.
Die Abtastrate des Datenloggers soll auf $\Delta t_{GLX} = 5\, \si{\second}$ eingestellt werden.
Die mit der Grundplatte verbundene Spannungsquelle soll auf $U_P = 5\, \si{\volt}$ eingestellt 
werden. Wenn die Isolierung auf die Stäbe gelegt wurde, kann der Schalter an der Grundplatte auf
'Heat' umgelegt und die Messung gestartet werden. Die Messung soll beendet werden, wenn die 
Temperatur an Thermoelement $T_7$ ungefähr 45°C beträgt. Ist dieser Punkt erreicht, so sollen 
die Isolierungen abgenommen werden und der Schalter an der Grundplatte soll von 'Heat' auf
'Cool' umgelegt werden. Die aufgenommenen Daten, welche auf dem Datenlogger gespeichert wurden,
sollen grafisch ausgewertet werden. Über den Temperaturverlauf an den Thermoelementen soll 
die Wärmeleitfähigkeit der Metalle ermittelt werden. Erst wenn die Temperatur aller Thermoelemente unter 30°C 
liegt kann mit der zweiten Messung fortgefahren werden.


\begin{figure}[H]
    \centering
    \includegraphics[width=10cm]{content/204.jpg}
    \label{fig:durchfuehrungbild2}
    \caption{NAMENAMENAMENAME}
\end{figure}


\subsection{Dynamische Methode}

Im Gegensatz zu der statischen Methode sollen die Stäbe nun periodisch erhitzt und 
abgekühlt werden. Die Abtastrate am Datenlogger soll auf $\Delta t_{GLX} = 1\, \si{\second}$
und die Spannung der Spannungsquelle auf $U_P = 8\, \si{\volt}$ eingestellt werden. Die Messung soll
erst gestartet werden, wenn alle Thermoelemente weniger als 30°C anzeigen. Auch hier sollen während 
der Messung die Isolierung auf die Stäbe gelegt werden (siehe \ref{durchfuehrungbild2}). Die erste Messung soll bei einer Periodendauer
von $T = 80\, \si{\second}$ durchgeführt werden, das heißt, dass der Schalter nach jeweils 
$40\, \si{\second}$ umgeschaltet werden soll. Es sollen mindestens sechs Perioden 
aufgenommen werden.
Nachdem die Messung beendet wurde, sollen die Stäbe wieder bis auf unter 30°C gekühlt werden.
Während des Kühlvorgangs können die Isolierungen abgenommen werden. Nachdem sich die Stäbe 
abgekühlt haben, kann eine zweite Messung durchgeführt werden. Diese unterscheidet sich von der
ersten dynamischen Messung dahingehend, dass die Periodendauer auf $T = 200\, \si{\second}$ 
erhöht wird.  
